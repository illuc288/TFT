\subsection{Invertible field theories and stable homotopy theory \extra}\label{Spectra Phases of Matter}

These theories can be studied with stable homotopy theory, in the sense that they are maps of spectra.

 Stable homotopy homotopy theory is the branch
of homotopy theory that studies phenomena and structures that are stable, i.e. that can occur in any 
dimension, or in any sufficiently large dimension independently of the exact dimension. The tool used to
reach higher dimensions is very often suspension\footnote{We define it in the appendix \ref{WhatSpectra}.}, this is 
why sometimes stable homotopy theory is characterized as the phenomena that are stable under
 suspension.

The investigation of stable phenomena is achieved via spectra, i.e. sequence of pointed spaces with maps
relating them one another. For instance the stable homotopy group of the sphere are exactly the
 homotopy groups of the sphere spectrum. See in the appendix \ref{WhatSpectra} for a fast paced 
modern introduction to spectra.

In what follows we sketch why invertible field theories are maps of spectra, this is however very
handwavy, for a more detailed sketch, look in the appendix.

The key to realizing why invertible field theories are maps of spectra
and thus computable via stable homotopy theory is the following theorem, due to May, which roughly
 says\footnote{See\ref{RecognitionConnectiveSpectra} for a precise statement.}
\begin{thm}[Recognition theorem for connective spectra]\label{RecognitionConnective2}
Connective spectra and Picard $\infty$-groupoids coincide, they are the same thing.
\end{thm}
Since stable homotopy theory involves $\infty$-categorical machinery, one must fully extend topological field theories
 (see \ref{RemarkExtendedTFTs}), i.e. with $(\infty,n)$-categories of bordisms as a source.
With the homotopy coherent additional structure of $(\infty,n)$-categories one gets that $|\Bord|$ and
  $\cat^\times$ are $\infty$-groupoids since
\begin{enumerate}
    \item we get $|\Bord|$ by adjoining all inverses for all morphisms and thus all morphisms become invertible
    \item  $\cat^\times$ is obtained by forgetting about non-invertible morphisms (and objects) of $\cat$ and thus we are left only with invertible morphisms
\end{enumerate} 

However, we do not have just $\infty$-groupoids but also a symmetric monoidal structure with
 duals on both $|\Bord|$ and $\cat^\times$. In short, they are Picard $\infty$-groupoid. Thanks to 
 \ref{RecognitionConnective2} we know that they are connective
  spectra.
 
 This is why $\tilde\Zf$ is not just a map of spaces but a map of spectra, and this enables the application of stable homotopy theory, making life easier. We can conclude this interlude on stable homotopy theory and invertible TFTs with the following motto
$$\text{\textit{Invertible TFTs are maps of connective spectra!}}$$
The inquiring reader might rightly ask: why is this good news? 
%THIS IS HOW I made sense of this, not sure if this is the reason, please correct!!! /ANDRE
The answer is that sadly we do not know much stuff about $(\infty,n)$-categories, e.g. we do not
 know well how co/limits 
work\footnote{However, some people are fortunately working on this, see \cite{moser2023inftynlimits}.}, but in turn
stable homotopy theory has been much more investigated and so we can understand them better than the usual
extended TFT.

This is not only of pure mathematical interest but it also allows the application of topological field theories to condensed
 matter
 theory! Classical phases of matter are governed by Landau
 theory of phase transition. However, the phases of a quantum material do not behave in the
  same manner. Fortunately, there are ways to describe them, or at least specific types of quantum
   matter (with gapped Hamiltonians for instance)\footnote{Kitaev for instance proposed in
     \cite{Kitaev_2009} that the topological insulators and superconductors of a certain type of condensed
     matter are classifiable via K-theory\footnote{See
         \url{https://ncatlab.org/nlab/show/topological+phase+of+matter} and
          \url{https://ncatlab.org/nlab/show/K-theory+classification+of+topological+phases+of+matter}.} and
           this was further refined via twisted equivariant K-theory by Dan Freed and Greg Moore in
            \cite{Freed_2013}. }. The topological field theories describing the
             (\href{https://ncatlab.org/nlab/show/topological+phase+of+matter}{topological}) phases of
              matter of some condensed matter systems (with short range entanglement) are
               \emph{invertible} field theories. The connection to stable homotopy theory allows to compute
                topological invariants of spectra and thereby of phases of matter. This was done by Freed and
                 Hopkins in \cite{Freed_2021} and in \cite{freed2014shortrange}.

