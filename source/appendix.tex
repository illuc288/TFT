\chapter{Infinity categories \extra}
\warning This is WIP, but the section on homotopy co/limits should already make sense.
% I plan to sketch how \infty-categories work, probably via a synthetic viewpoint
%Moreover, via such homotopy co/limits one can define other fundamental notions for spectra,
%so in what follows, we try to convey an idea for how they work.
\section{Homotopy co/limits \extra}
This preamble is based on \cite{HarpazAlgebras}.

An adoptable perspective
 to do (topological\footnote{Topology is the motherland of homotopy theory, however it is travelling
to other places such as representation theory, e.g. \cite{stroppel2022categorification} or Gaitsgory-Rozenblyum's
work,
 or algebraic geometry, e.g. derived algebraic geometry of Toen, Vezzosi and Lurie.}) homotopy theory is working
in the homotopy category of topological spaces\footnote{By 'topological spaces' we mean nice ones, i.e. CW-complexes
or similar ones, for instance compactly genereated weakly Hausdorff.}
 $\operatorname{hTop}$, that is the category of topological
spaces and continuous maps between them up to homotopy. 
This has however
several drawbacks:
\begin{itemize}
    \item we can intuit that some information gets lost, where are all the higher homotopies between homotopies?
    \item it is not bicomplete, i.e. it does not have all co/limits, oppositely to Top, the not-homotopy-category of topological
    spaces
    \item the co/limits that interest us, i.e. homotopy co/limits, are not determined by universal properties
\end{itemize}
A revolutionary way to do homotopy theory
 is to avoid working in the hTop and work in the $\infty$-category of topological spaces.
This category is bicomplete and we can phrase the desired homotopy co/limits via the universal properties 
phrasable with the underlying categorical structure. 
This strategy to pass to $\infty$-categories generalize in 2 ways:
\begin{itemize}
    \item one can talk about homotopy co/limits also in other categories
    \item all diagrams commute homotopy coherently, not only the ones describing co/limits
\end{itemize}
Thanks to this new version of commutativity, we can do powerful stuff, for instance defining $\E_n$-algebras, as
we did in \ref{EAlg}. 

Now we sketch more precisely how such homotopy and non-homotopy co/limits work, but see 
\cite[2.6]{HarpazAlgebras} for a much more detailed summary and
\cite[Section 4]{land2021introduction} for a thorough introduction.
Very generally, homotopy co/limits are objects making certain diagrams commute via homotopies, i.e. homotopy
 coherently and with certain
universal properties.  Every co/limit in an $\infty$-category works as a homotopy
co/limit.
To sensibly talk about this, we first need to define 
what are co/limits in ordinary categories. Fortunately, they work similarly.
\begin{defn}[Terminal, initial, zero objects]
    A terminal object in a category $\cat$ is an object $\ast\in\cat$ such that for every
    $X\in\cat$ there is a unique morphism $X\to\ast$
    
    An initial object in a category $\cat$ is an object $\emptyset\in\cat$ such that for every
    $X\in\cat$ there is a unique morphism $\emptyset\to X$
    
    A zero object is an object that is both initial and final. A category is called pointed if it has a zero object.
\end{defn}

\begin{defn}[Diagram]
    Until now, we spoke of diagrams informally, there is however a precise definition. Given a
    small\footnote{This means that the collection of objects and morphisms are both sets}
    category $\mathscr{J}$, a diagram of 
    shape $\mathscr{J}$ in $\cat$ is a functor $D:\mathscr{J}\to \cat$
\end{defn}
\begin{defn}[Co/limits]
    Given a category $\mathscr{J}$ we can construct 
    \begin{itemize}
        \item the cone\footnote{Aka left cone} of $\mathscr{J}$: $\mathscr{J}^{\lhd}$ by adding an initial object to $\mathscr{J}$
        \item the cocone\footnote{Aka right cone} of $\mathscr{J}$: $\mathscr{J}^{\rhd}$ by adding a terminal object to $\mathscr{J}$
    \end{itemize}
    Given a diagram $D$ of shape $\mathscr{J}$ in $\cat$ and $X\in\cat$ we have 
    
    \begin{itemize}
        \item a cone over $D$ that is a natural transformation $\gamma: X\Rightarrow D$, where $X$ denotes the functor sending every object in $\mathscr{J}$ to $X$. So, because of naturality
        of $\gamma$ for $M\xrightarrow{f}N$ in $\mathscr{J}$ there is the following commutative diagram:
        % https://q.uiver.app/#q=WzAsMyxbMSwwLCJYKE4pPVg9WChNKSJdLFswLDEsIkQoTikiXSxbMiwxLCJEKE0pIl0sWzAsMSwiXFxnYW1tYV97Tn0iLDJdLFswLDIsIlxcZ2FtbWFfe019Il0sWzEsMiwiRChmKSIsMl1d
        \[\begin{tikzcd}[cramped]
            & {X(N)=X=X(M)} \\
            {D(N)} && {D(M)}
            \arrow["{\gamma_{N}}"', from=1-2, to=2-1]
            \arrow["{\gamma_{M}}", from=1-2, to=2-3]
            \arrow["{D(f)}"', from=2-1, to=2-3]
        \end{tikzcd}\]
        We have then also the category of cones over $D$. 
        An equivalent description of a cone over $D$ is a functor $\mathscr{J}^{\lhd}\to\cat$
        hence the category of cones is the functor category
        $\Fun_D(\mathscr{J}^{\lhd},\cat)$.
        
        \item a cocone over $D$ that is a natural transformation $\eta: D\Rightarrow X$ and because of naturality
        of $\gamma$ for $M\xrightarrow{f}N$ in $\mathscr{J}$ there is the following commutative diagram:
        % https://q.uiver.app/#q=WzAsMyxbMSwxLCJYKE4pPVg9WChNKSJdLFswLDAsIkQoTikiXSxbMiwwLCJEKE0pIl0sWzEsMCwiXFxldGFfe059IiwyXSxbMiwwLCJcXGV0YV97TX0iXSxbMSwyLCJEKGYpIl1d
        \[\begin{tikzcd}[cramped]
            {D(N)} && {D(M)} \\
            & {X(N)=X=X(M)}
            \arrow["{\eta_{N}}"', from=1-1, to=2-2]
            \arrow["{\eta_{M}}", from=1-3, to=2-2]
            \arrow["{D(f)}", from=1-1, to=1-3]
        \end{tikzcd}\]
        We have then also the category of cocones over $D$. An equivalent description of a cocone over $D$ is a functor $\mathscr{J}^{\rhd}\to\cat$
        hence the category of cocones is the functor category
        $\Fun_D(\mathscr{J}^{\rhd},\cat)$
    \end{itemize}
    Then, 
    \begin{itemize}
        \item the limit of the diagram $D$ is a terminal object in the category of cones over $\mathscr{J}$, i.e. $\Fun_D(\mathscr{J}^{\lhd},\cat)$
        \item the colimit of the diagram $D$ is an initial object in the category of cocones over $\mathscr{J}$, i.e. $\Fun_D(\mathscr{J}^{\rhd},\cat)$
    \end{itemize}
    In order to get a definition of homotopy co/limit or $(\infty,1)$-co/limit, just substitute your favourite
    model of $\infty$-category with 'category' in this definition. 
\end{defn}
\begin{ex}[Cartesian product]
    We previously defined the cartesian product of two objects by saying that some diagram commutes and the product  has a certain universal property (\ref{CartProd}). 
    Such cartesian product is a limit over a diagram shaped as a discrete category with two objects.
    Let us name this discrete category with two different objects $\mathscr{J}=\{0,1\}$, discrete means that
    there are only the identity morphisms $id_0$ and $id_1$. 
    Given $\cat$ an arbitrary category, observe a diagram 
    $$D:\mathscr{J}\to \cat $$ 
    $$0\mapsto A$$
    $$1\mapsto B$$
    Then, a cartesian product $A\times B$ is the limit over $D$
    % https://q.uiver.app/#q=WzAsNSxbMSwwLCJZIl0sWzIsMSwiQiJdLFswLDBdLFswLDEsIkEiXSxbMSwxLCJBXFx0aW1lcyBCIl0sWzAsMSwiZyJdLFs0LDMsInByXzEiXSxbMCwzLCJmIiwyXSxbNCwxLCJwcl8yIiwyXSxbMCw0LCJcXGV4aXN0cyEgdSIsMSx7InN0eWxlIjp7ImJvZHkiOnsibmFtZSI6ImRhc2hlZCJ9fX1dXQ==
    \[\begin{tikzcd}
        {} & Y \\
        A & {A\times B} & B
        \arrow["g", from=1-2, to=2-3]
        \arrow["{pr_1}", from=2-2, to=2-1]
        \arrow["f"', from=1-2, to=2-1]
        \arrow["{pr_2}"', from=2-2, to=2-3]
        \arrow["{\exists! u}"{description}, dashed, from=1-2, to=2-2]
    \end{tikzcd}\]
    The two projections exist because $A\times B$ is a cone: a cone is an initial object in the shape category $\mathscr{J}$ and $D$ is a functor, so it imports the unique morphisms from the initial object to where $0$ and $1$ are mapped, i.e. to $A$ and $B$.
    
    \noindent $u$ exists and is unique because $A\times B$ is the terminal object in the category of cones.
    
    \noindent The two smaller triangles commute because of the naturality of the morphism $u$: recall that
    the category of cones is a functor category hence morphisms between cones 
    are natural transformations and note that $pr_1$ and $f$ for instance are images of morphisms in
    $\mathscr{J}^{\lhd}$
\end{ex}
We already sketched what might mean that a diagram is homotopy coherently commutative in the
case of an $\mathbb{E}_1$-algebra (\ref{EAlg}). For the sake of clarity we sketch it here again,
in a more general setting. Recall that all morphisms higher than 1 in an $\infty$-category are weakly invertible.
% https://q.uiver.app/#q=WzAsNSxbMCwwLCJBIl0sWzAsMiwiQiJdLFsyLDIsIkQiXSxbMiwwLCJDIl0sWzEsMSwiLi4uIl0sWzAsMywiZyJdLFswLDEsImYiLDJdLFsxLDIsInEiLDJdLFszLDIsInIiXSxbMSwzLCJoIiwwLHsiY3VydmUiOi0zLCJzaG9ydGVuIjp7InNvdXJjZSI6MTAsInRhcmdldCI6MTB9LCJsZXZlbCI6Mn1dLFszLDEsImsiLDAseyJjdXJ2ZSI6LTMsInNob3J0ZW4iOnsic291cmNlIjoxMCwidGFyZ2V0IjoxMH0sImxldmVsIjoyfV0sWzksMTAsIm0iLDIseyJjdXJ2ZSI6Mywic2hvcnRlbiI6eyJzb3VyY2UiOjIwLCJ0YXJnZXQiOjIwfX1dLFsxMCw5LCJsIiwyLHsiY3VydmUiOjMsInNob3J0ZW4iOnsic291cmNlIjoyMCwidGFyZ2V0IjoyMH19XV0=
\[\begin{tikzcd}[cramped]
    A && C \\
    & {...} \\
    B && D
    \arrow["g", from=1-1, to=1-3]
    \arrow["f"', from=1-1, to=3-1]
    \arrow["q"', from=3-1, to=3-3]
    \arrow["r", from=1-3, to=3-3]
    \arrow[""{name=0, anchor=center, inner sep=0}, "h", curve={height=-18pt}, shorten <=7pt, shorten >=7pt, Rightarrow, from=3-1, to=1-3]
    \arrow[""{name=1, anchor=center, inner sep=0}, "k", curve={height=-18pt}, shorten <=7pt, shorten >=7pt, Rightarrow, from=1-3, to=3-1]
    \arrow["m"', curve={height=18pt}, shorten <=8pt, shorten >=8pt, Rightarrow, from=0, to=1]
    \arrow["l"', curve={height=18pt}, shorten <=8pt, shorten >=8pt, Rightarrow, from=1, to=0]
\end{tikzcd}\]
This diagram does not commute strictly, but via homotopies.
This means that $h:q\circ f\simeq r\circ g:k$, instead of what we are used to: $q\circ f= r\circ g$.
Not only this however, it is the case that $k\circ h\simeq id_{q\circ f}$ and $h\circ k\simeq id_{r\circ g}$
, instead of the usual strict invertibility, and thus via the 3-morphisms $l$ and $m$, the 2-morphisms 
$k$ and $h$ are homotopic, i.e. $l:k\simeq h:m$. Then there
will be some explicit 4-morphisms witnessing that $m\simeq l$, and so on to $\infty$.

%I am unsure about the sketch above!!!! Please check /Andrea


\begin{ex}[Homotopy pullback/$(\infty,1)$-pullback]
    Let  $\mathscr{J}$ be an $\infty$-category with objects and 1-morphisms $0\to1\leftarrow2$ and
    $D:\mathscr{J}\to \cat$ be a diagram. To be rigorous, we would need to define what is an
    $\infty$-functor. However, since this necessitates getting our hands dirty with a model,
    we prefer to remain loose: an $\infty$-functor is an assignment that respects composition and identity
    morphisms\footnote{i.e. is functorial.} not only on 1-morphisms, but for also all higher invertible morphisms.
    Then, an $\infty$-pullback of 
    is the homotopy limit of $D$.
    
    We now unpack what this means. A cone of $\mathscr{J}$, where $\emptyset$ is the new initial object
    looks like this:
    % https://q.uiver.app/#q=WzAsNCxbMCwwLCJcXGVtcHR5c2V0Il0sWzAsMSwiMCJdLFsxLDEsIjEiXSxbMSwwLCIyIl0sWzAsMV0sWzEsMl0sWzAsM10sWzMsMl0sWzAsMl0sWzgsMSwiIiwxLHsic2hvcnRlbiI6eyJzb3VyY2UiOjIwfSwic3R5bGUiOnsidGFpbCI6eyJuYW1lIjoiYXJyb3doZWFkIn19fV0sWzgsMywiIiwxLHsic2hvcnRlbiI6eyJzb3VyY2UiOjIwfSwic3R5bGUiOnsidGFpbCI6eyJuYW1lIjoiYXJyb3doZWFkIn19fV1d
    \[\begin{tikzcd}[cramped]
        \emptyset & 2 \\
        0 & 1
        \arrow[from=1-1, to=2-1]
        \arrow[from=2-1, to=2-2]
        \arrow[from=1-1, to=1-2]
        \arrow[from=1-2, to=2-2]
        \arrow[""{name=0, anchor=center, inner sep=0}, from=1-1, to=2-2]
        \arrow[shorten <=2pt, Rightarrow, 2tail reversed, from=0, to=2-1]
        \arrow[shorten <=2pt, Rightarrow, 2tail reversed, from=0, to=1-2]
    \end{tikzcd}\]
    Where the two small triangles commute via higher morphisms denoted with $\Leftrightarrow$.
    
    Let $D(1)=X$, $D(2)=Y$ and $D(0)=Z$. Then, the $\infty$-pullback of $D$ is the terminal object 
    in the $\infty$-functor $\infty$-category $\Fun_D(\mathscr{J},\cat)$. Denote the image of $\emptyset$ of such pullback with $Y\times_X Z$. Then there is the following homotopy coherent diagram
    % https://q.uiver.app/#q=WzAsNCxbMCwwLCJZXFx0aW1lc19YIFoiXSxbMCwxLCJZIl0sWzEsMSwiWiJdLFsxLDAsIlgiXSxbMCwxXSxbMSwyXSxbMCwzXSxbMywyXSxbMCwyXSxbOCwxLCIiLDEseyJzaG9ydGVuIjp7InNvdXJjZSI6MzAsInRhcmdldCI6MTB9LCJzdHlsZSI6eyJ0YWlsIjp7Im5hbWUiOiJhcnJvd2hlYWQifX19XSxbOCwzLCIiLDEseyJzaG9ydGVuIjp7InNvdXJjZSI6MzAsInRhcmdldCI6MTB9LCJzdHlsZSI6eyJ0YWlsIjp7Im5hbWUiOiJhcnJvd2hlYWQifX19XV0=
    \[\begin{tikzcd}[cramped]
        {Y\times_X Z} & X \\
        Y & Z
        \arrow[from=1-1, to=2-1]
        \arrow[from=2-1, to=2-2]
        \arrow[from=1-1, to=1-2]
        \arrow[from=1-2, to=2-2]
        \arrow[""{name=0, anchor=center, inner sep=0}, from=1-1, to=2-2]
        \arrow[shorten <=4pt, shorten >=1pt, Rightarrow, 2tail reversed, from=0, to=2-1]
        \arrow[shorten <=4pt, shorten >=1pt, Rightarrow, 2tail reversed, from=0, to=1-2]
    \end{tikzcd}\]
    Because the $\infty$-pullback is a functor and thus preserves the homotopy-coherent commutativity
    of diagrams in the shape $\mathscr{J}$, by preserving compositionality for all $n$-morphisms.
    Since it is the terminal object in the category of cones over the diagram $D$ for every other cone
    mapping $\emptyset$ to an object $Q\in\cat$, then there is a unique morphism $Q\to Y\times_X Z$
    making the following diagram commute homotopy coherently
    
    % https://q.uiver.app/#q=WzAsNSxbMSwxLCJZXFx0aW1lc19YIFoiXSxbMSwyLCJZIl0sWzIsMiwiWiJdLFsyLDEsIlgiXSxbMCwwLCJRIl0sWzAsMV0sWzEsMl0sWzAsM10sWzMsMl0sWzAsMl0sWzQsMF0sWzQsMV0sWzQsM10sWzksMSwiIiwxLHsic2hvcnRlbiI6eyJzb3VyY2UiOjMwLCJ0YXJnZXQiOjEwfSwic3R5bGUiOnsidGFpbCI6eyJuYW1lIjoiYXJyb3doZWFkIn19fV0sWzksMywiIiwxLHsic2hvcnRlbiI6eyJzb3VyY2UiOjMwLCJ0YXJnZXQiOjEwfSwic3R5bGUiOnsidGFpbCI6eyJuYW1lIjoiYXJyb3doZWFkIn19fV0sWzEwLDExLCIiLDAseyJzaG9ydGVuIjp7InNvdXJjZSI6MjAsInRhcmdldCI6MjB9LCJzdHlsZSI6eyJ0YWlsIjp7Im5hbWUiOiJhcnJvd2hlYWQifX19XSxbMTAsMTIsIiIsMix7InNob3J0ZW4iOnsic291cmNlIjoyMCwidGFyZ2V0IjoyMH0sInN0eWxlIjp7InRhaWwiOnsibmFtZSI6ImFycm93aGVhZCJ9fX1dXQ==
    \[\begin{tikzcd}[cramped]
        Q \\
        & {Y\times_X Z} & X \\
        & Y & Z
        \arrow[from=2-2, to=3-2]
        \arrow[from=3-2, to=3-3]
        \arrow[from=2-2, to=2-3]
        \arrow[from=2-3, to=3-3]
        \arrow[""{name=0, anchor=center, inner sep=0}, from=2-2, to=3-3]
        \arrow[""{name=1, anchor=center, inner sep=0}, from=1-1, to=2-2]
        \arrow[""{name=2, anchor=center, inner sep=0}, from=1-1, to=3-2]
        \arrow[""{name=3, anchor=center, inner sep=0}, from=1-1, to=2-3]
        \arrow[shorten <=4pt, shorten >=1pt, Rightarrow, 2tail reversed, from=0, to=3-2]
        \arrow[shorten <=4pt, shorten >=1pt, Rightarrow, 2tail reversed, from=0, to=2-3]
        \arrow[shorten <=2pt, shorten >=2pt, Rightarrow, 2tail reversed, from=1, to=2]
        \arrow[shorten <=4pt, shorten >=4pt, Rightarrow, 2tail reversed, from=1, to=3]
    \end{tikzcd}\]
    The two small triangles formed by morphisms that have $Q$ as source and by the projections of $Y\times_X Z$ (the ones on the upper left) commute because of the naturality of morphisms in 
    $\Fun_D(\mathscr{J}^{\lhd},\cat)$
\end{ex}
\begin{notat}
    From now on, when we are talking about commutative diagrams in an $\infty$-category, we do not write the double arrows $\Leftrightarrow$ symbolizing the 
    higher morphisms that make the diagram commute.
\end{notat}
There is the dual notion of homotopy pullback: the homotopy pushout, which we now use to define
suspensions in the next chapter of the appendix.
\chapter{What are spectra? \extra}\label{WhatSpectra}
This subsection is based off \cite{Luriealgebra}, \cite{AdamsLoops}, \cite{GregoricSpectra},
\cite{Mazel-Gee2024}, \cite{NardinStable} and \cite{CalleInfiniteLoops}.

Stable homotopy homotopy theory is the branch
of homotopy theory that studies phenomena and structures that are stable, i.e. that can occur in any 
dimension, or in any sufficiently large dimension independently of the exact dimension. The tool used to
reach higher dimensions is very often suspension\footnote{We define it in the next subsection.}, this is 
why sometimes stable homotopy theory is characterized as the phenomena that are stable under
suspension.

The paradigmatic example of stable phenomena are
the stable homotopy groups of the sphere are the homotopy groups of the sphere $\pi_{n+i}(S^n)$ such
that $n>i+1$. They are stable because thanks to Freudenthal's suspension theorem implies that the 
suspension functor\footnote{Do not fear if you do not know what it is yet, we define it in the next
    subsection.}
$S^n\xrightarrow{\Sigma} S^{n+1}$
induces
an isomorphism on certain homotopy groups of the sphere: the stable ones, i.e. where $n>i+1$
$$ \pi_{n+i}(S^n)\cong  \pi_{n+i+1}(S^{n+1})$$

The investigation of stable phenomena is achieved via spectra, i.e. sequence of pointed spaces with maps
relating them one another. For instance the stable homotopy group of the sphere are exactly the
homotopy groups of the sphere spectrum. This begs the question: what are spectra?

We already have seen some spectra and definitions of some kinds of spectra in \ref{homotopyCob}, e.g. the 
sphere spectrum and the suspension spectrum. In that section, we defined suspension as a quotient.
However, one can define a suspension in full generality, i.e. for any pointed $\infty$-category, via certain homotopy
colimits. 

\begin{defn}[Suspension]
    Let $\cat$ be a pointed\footnote{A category with a zero object: an object that is both initial and final.} 
    $\infty$-category with the zero object denoted by 0 
    and let $X\in\cat$. Then the suspension $\Sigma X$ of $X$ is the pushout 
    % https://q.uiver.app/#q=WzAsNCxbMCwwLCJYIl0sWzAsMSwiMCJdLFsxLDAsIjAiXSxbMSwxLCJcXFNpZ21hIFgiXSxbMCwxXSxbMCwyXSxbMSwzXSxbMiwzXSxbMywwLCIiLDEseyJzdHlsZSI6eyJuYW1lIjoiY29ybmVyIn19XV0=
    \[\begin{tikzcd}[cramped]
        X & 0 \\
        0 & {\Sigma X}
        \arrow[from=1-1, to=2-1]
        \arrow[from=1-1, to=1-2]
        \arrow[from=2-1, to=2-2]
        \arrow[from=1-2, to=2-2]
        \arrow["\lrcorner"{anchor=center, pos=0.125, rotate=180}, draw=none, from=2-2, to=1-1]
    \end{tikzcd}\]
\end{defn}

\begin{defn}[Suspension functor]
    Given a pointed $\infty$-category $\cat$ the assignment $X\mapsto \Sigma X$ gives rise to a functor
    $\Sigma:\cat\to\cat$
\end{defn}
Dually, one can define the loop space of an object
\begin{defn}[Loop space]
    Let $\cat$ be a pointed
    $\infty$-category and let $X\in\cat$. Then the loop space $\Omega X$ is the pullback 
    % https://q.uiver.app/#q=WzAsNCxbMCwwLCJcXE9tZWdhIFgiXSxbMCwxLCIwIl0sWzEsMCwiMCJdLFsxLDEsIlgiXSxbMCwxXSxbMCwyXSxbMSwzXSxbMiwzXSxbMCwzLCIiLDEseyJzdHlsZSI6eyJuYW1lIjoiY29ybmVyIn19XV0=
    \[\begin{tikzcd}[cramped]
        {\Omega X} & 0 \\
        0 & X
        \arrow[from=1-1, to=2-1]
        \arrow[from=1-1, to=1-2]
        \arrow[from=2-1, to=2-2]
        \arrow[from=1-2, to=2-2]
        \arrow["\lrcorner"{anchor=center, pos=0.125}, draw=none, from=1-1, to=2-2]
    \end{tikzcd}\]
\end{defn}
\begin{defn}[Loop space functor]
    Given a pointed $\infty$-category $\cat$ the assignment $X\mapsto \Omega X$ gives rise to a functor $\Omega:\cat\to\cat$
\end{defn}
\begin{rem}
    One very important property of the loop space functor is the shift in homotopy groups:
    $$ \pi_n(\Omega X)=\pi_{n+1}(X)$$
    If you took a class on algebraic topology, you most probably already witnessed this phenomenon, 
    the fundamental group is defined as
    $$ \pi_0(\Omega X)=\pi_1(X)$$
\end{rem}
\begin{rem}
    As already mentioned in \ref{EAlg}, a loop space is an example of $\mathbb{E}_1$-space and 
    $n$-iterated loop spaces are $\mathbb{E}_n$-spaces by the Dunn additivity theorem.
\end{rem}
\begin{defn}[Alternative definition of adjunction]\label{HigherAdjFun}
    We defined adjoint functors via the natural tranformations unit and counit (\ref{AdjFun}). There is however an equivalent
    formulation:
    two $\infty$-functors $\cat\underset{L}{\overset{R}{\leftrightarrows}}\dat$ form an adjunction $L\dashv R$, i.e. where $L$ is the left adjoint
    and $R$ is the right adjoint, if between the two Hom-$\infty$-functors 
    $$\Hom_\cat(L(-),-):\cat^{\operatorname{op}}\times\cat \to \operatorname{Grpd}_\infty$$
    and
    $$\Hom_\dat(-,R(-)):\dat^{\operatorname{op}}\times\dat\to \operatorname{Grpd}_\infty$$
    there is a natural isomorphism $\Hom_\cat(L(-),-)\simeq \Hom_\dat(-,R(-))$
    
    This definition is easily derivable from the one with units we gave previously, given 
    the unit $\epsilon:id_{\dat}\Rightarrow R\circ L$ and $X\in\cat$ and $Y\in\dat$, 
    the following holds because of one of the triangle identities: 
    % https://q.uiver.app/#q=WzAsMyxbMCwwLCJcXEhvbV97XFxjYXR9KEwoWSksWCkiXSxbMSwxLCJcXEhvbV9cXGRhdChSKEwoWSkpLFIoWCkpIl0sWzIsMCwiXFxIb21fXFxkYXQoWSxSKFgpKSJdLFswLDFdLFsxLDJdLFswLDIsIlxcc2ltZXEiXV0=
    \[\begin{tikzcd}[cramped]
        {\Hom_{\cat}(L(Y),X)} && {\Hom_\dat(Y,R(X))} \\
        & {\Hom_\dat(R(L(Y)),R(X))}
        \arrow[from=1-1, to=2-2]
        \arrow[from=2-2, to=1-3]
        \arrow["\simeq", from=1-1, to=1-3]
    \end{tikzcd}\]
\end{defn}
\begin{rem}[Suspension-loop adjunction]
    There is an adjunction $\Sigma\dashv\Omega$ and thus for any pointed $\infty$-category
    $\cat$ it holds that $\Hom_{\cat}(\Sigma-,-)\simeq \Hom_{\cat}(-,\Omega-)$. If it is not sufficiently clear from
    the fact that 
    $\Omega X=0\times_X 0$ and $\Sigma X=0\amalg_X 0$, do not worry. In the case that it interests us the most
    it can be seen from another perspective. Recall that for $X\in \Top_\ast$
    $\Sigma X= X\land S^1$ and $\Omega_x=\operatorname{Map}_{\ast}(S^1,X)$ and that $\Top_\ast$ is cartesian
    monoidal closed, i.e. for any $A,B,C\in\Top_\ast$ it holds that $\Hom_{\Top_{\ast}}(A\land B, C)\cong\Hom_{\Top_{\ast}}(A,\operatorname{Map}_{\ast}(B,C))$ and hence 
    $\Hom_{\Top_{\ast}}(-\land S^1,-)\cong\Hom_{\Top_{\ast}}(-,\operatorname{Map}_{\ast}(S^1,-))$
\end{rem}
\begin{defn}[($\Omega$-)Spectrum\footnote{It is denoted $\Omega$-spectrum in the older literature, e.g. books or 
        articles by May or Adams like \cite{AdamsLoops}.}]
    A spectrum is a collection of 
    pointed spaces $\{E_n\}_{n\in\Z}$ with equivalences $\delta_n:E_n\simeq \Omega E_{n+1}$
\end{defn}
\begin{rem}
    Note that via the adjunction $\Sigma\dashv\Omega$ we can get a prespectrum from a spectrum:
    given $X_i,X_{i+1}\in\sat_\ast$ it holds that 
    $\Hom_{\sat_{\ast}}(X_i,\Omega X_{i+1})\simeq \Hom_{\sat_{\ast}}(\Sigma X_i,X_{i+1})$ and thus we can 
    get the pointed maps we need in order to define a prespectrum. This means that every spectrum
    is a prespectrum, i.e. the category of spectra is a subcategory of the category of prespectra. This is the
    reason why the synonym of prespectra 'sequential spectra' is counterintuitive.
\end{rem}


\begin{defn}[$\infty$-category of spectra]
    The $\infty$-category of spectra is the category where the objects are spectra. %The fact that there
    %are equivalences $X_i\simeq\Omega X_{i+1}$ 
    
    
    This is equivalent to saying that the category of spectra is the sequential homotopy limit 
    $$\operatorname{lim}(...\leftarrow\sat_\ast\leftarrow\sat_\ast\leftarrow\sat_\ast\leftarrow ...)$$
    where we can see the diagram we are taking the limit over as a functor 
    $D:\Z^{\operatorname{op}}\to\infty\operatorname{-Cat}$ where $\Z$ is considered as a category
    via its ordering ($\leq$ corresponds to a morphism $\to$) and for each $i\in\Z$ $D(i)=\sat_\ast$. 
    %Sometimes 
    %such diagrams are called towers. %TBD
    
    From this it follows that a map of spectra $f:\{X_i\}_{i\in\Z}\to\{Y_i\}_{i\in\Z}$ is given by components 
    that are pointed maps $f_i:X_i\to Y_i$ such that the following square commutes homotopy coherently
    % https://q.uiver.app/#q=WzAsNCxbMCwwLCJYX2kiXSxbMSwwLCJZX2kiXSxbMCwxLCJcXE9tZWdhIFhfe2krMX0iXSxbMSwxLCJcXE9tZWdhIFlfe2krMX0iXSxbMCwxLCJmX2kiXSxbMCwyLCJcXHNpbWVxIiwyLHsic3R5bGUiOnsiaGVhZCI6eyJuYW1lIjoibm9uZSJ9fX1dLFsxLDMsIlxcc2ltZXEiLDAseyJzdHlsZSI6eyJoZWFkIjp7Im5hbWUiOiJub25lIn19fV0sWzIsMywiXFxPbWVnYSBmX3tpKzF9IiwyXV0=
    \[\begin{tikzcd}[cramped]
        {X_i} & {Y_i} \\
        {\Omega X_{i+1}} & {\Omega Y_{i+1}}
        \arrow["{f_i}", from=1-1, to=1-2]
        \arrow["\simeq"', no head, from=1-1, to=2-1]
        \arrow["\simeq", no head, from=1-2, to=2-2]
        \arrow["{\Omega f_{i+1}}"', from=2-1, to=2-2]
    \end{tikzcd}\] %UNSURE ABOUT THIS; PLEAASE CHECK /ANDREA
\end{defn}
The initial object of the $\infty$-category of spectra is the sphere spectrum $\mathds{S}$.
\begin{notat}
    An $n$-connective space is a space with trivial homotopy groups for
    all $k< n$. An $n$-connective space is an $n-1$-connected space. A space is 0-connective if and only
    if it is non-empty.
\end{notat}
\begin{defn}[Infinite loop space]\label{InfiniteLoop}
    An infinite loop space is a non-empty pointed space $X_0$ with an infinite sequence of pointed
    spaces
    $X_0,X_1,X_2,...$ with weak equivalences for $n\geq 0$
    $$X_n\simeq \Omega X_{n+1}$$
    
    %\footnote{The loop space object is an inverse operation to the delooping, meaning that that given some monoidal category $\cat$, $\cat$ is the loop space object of the delooping of $\cat$. The name comes from the topological notion, see \url{https://en.wikipedia.org/wiki/Loop_space}
        
        %In general, loop space objects inherit the monoidal structure from the composition of loops. We would need to talk about homotopy pullbacks in order to strictly define what a loop space object is.}
    
\end{defn}

Infinite loop spaces seem very similar to what we defined as spectra ($\Omega$-spectra). 

% However, there is a crucial 
%difference: there is no space to start from in a spectrum, since there are deloopings for all spaces, i.e. $ X_{-1}\simeq \Omega X_{0}$, 
%$ X_{-2}\simeq \Omega X_{-1}$ and so on, while in an infinite loop space there are not.

\noindent In fact, we will soon see that infinite loop spaces coincide with a particular kind of spectra:
connective spectra.
\begin{defn}[Underlying (infinite loop) space functor]
    Given a spectrum, the assignment of a spectrum $ \{X_i\}_{i\in\Z}$ to its underlying
    infinite loop space, aka 0-th space, is a functor.
    $$\Omega^{\infty}:\operatorname{Sp}\to\sat_\ast$$   
    $$ \{X_i\}_{i\in\Z}\mapsto X_0$$
\end{defn}
Such assignment has a left adjoint.
\begin{defn}[Suspension spectrum functor]
    $$\Sigma^{\infty}:\sat_\ast\to\operatorname{Sp}$$
    $$X\mapsto \{\Sigma^i X\}_i$$
    sending pointed spaces to suspension spectra, i.e. sequences of spaces  $\{\Sigma^i X\}_i$ with pointed
    maps $\Sigma(\Sigma^i X)=\Sigma^{i+1} X$. Such suspension spectra are indeed spectra las we defined
    then (aka $\Omega$-spectra) thanks to the adjunction $\Sigma\dashv\Omega$.
    
    An example of such spectra is the sphere spectrum we already encountered in \ref{homotopyCob}. 
\end{defn}
%\begin{defn}[Classifying spectrum]
%Let $M$ be an $\mathbb{E}_\infty$-space, i.e. an $\mathbb{E}_\infty$-algebra in $\sat_\ast$ (see
% \ref{EAlg}). Then we define the classifying spectrum of $M$
%$$\mathbf{B}^{\infty}(M):=\{\mathbf{B}^n\}$$ %TBD
%It might be useful to check out what is a classifying space of a topological
%group\footnote{\url{https://en.wikipedia.org/wiki/Classifying_space}}.
%\end{defn}
%\begin{defn}[
%\end{defn}


\begin{thm}[Recognition theorem for $n$-iterated loop spaces, Boardman-Vogt, May]\label{RecognitionNLOOPS}
    For any $n> 0$, $n$-iterated loop spaces coincide with grouplike $\E_n$-spaces. 
    More rigorously: the 
    $n$-iterated loop space functor $\Omega^{n}:\sat_\ast \to\sat_\ast$ restricts to an equivalence of
    $\infty$-categories between $n$-connective pointed spaces and grouplike $\E_n$-spaces
    $$\Omega^n:\sat^{\geq n}_{\ast}\simeq \Alg_{\mathbb{E}_n}^{\operatorname{gp}}(\sat) $$
\end{thm}

See \cite[5.2.6.10]{Luriealgebra} for a rigorous proof. Note anyway that it follows from an easier fact
to prove (see \cite[Theorem 3.9]{NardinStable}):
\begin{thm}[Recognition theorem for loop spaces]
    Loop spaces coincide with grouplike $\E_1$-spaces. 
    More rigorously: the 
    loop space functor $\Omega:\sat_\ast \to\sat_\ast$ restricts to an equivalence of
    $\infty$-categories between 0-connective pointed spaces and grouplike $\E_1$-spaces
    
    $$\sat^{\geq 0}_{\ast}\overset{\Omega}{\simeq}\Alg_{\mathbb{E}_1}^{\operatorname{gp}}(\sat) $$
\end{thm}
Combining this with the Dunn additivity theorem (\ref{DunnAdds}) allows to prove the recognition theorem
for $n$-iterated loop spaces by induction. The base case: $n=1$ is the recognition theorem for loop
spaces and the inductive step holds via the Dunn additivity theorem.

As said in the preamble of this section, stable homotopy groups of a space $A\in\sat$, denoted 
$\pi^{s}_{i}(A)$, are exactly the homotopy groups of the spectrum of $A$, $\Sigma^{\infty}A$. Thus,
$\pi^{s}_{i}(A)\cong \pi_{i}(\Sigma^{\infty}A)$. There is however a natural question: what are the
homotopy groups of a spectrum? One interesting fact is that
while the degree homotopy groups for usual topological spaces are defined to be positive, homotopy
groups of spectra can be defined and be non-trivial also for negative degrees.
\begin{defn}[Homotopy groups of spectra]
    Given a prespectrum $\{ E_i\}_{i\in\Z}$ with pointed maps $\sigma_i:\Sigma E_i\to E_{i+1}$,
    for $n,k\in\Z$, the $n$th homotopy group of 
    $\{ E_i\}_{i\in\Z}$ is 
    $$\pi_n(\{ E_i\}_{i\in\Z}):=\underset{\rightarrow}{\operatorname{lim}}(...\to \pi_{n+k}(E_k)\xrightarrow{\Sigma} \pi_{n+k+1}(\Sigma E_{k})\xrightarrow{\pi_{n+k+1}(\sigma_k)}\pi_{n+k+1}(E_{k+1})\to ...)$$
    where all the $\pi_\ast$ inside the parentheses indicate the homotopy groups of the various pointed
    spaces $E_k\in \{ E_i\}_{i\in\Z}$. Since $k\in\Z$, there are indeed also homotopy groups with negative 
    degrees.
    
    If a prespectrum $\{ X_i\}_{i\in\Z}$ is also a spectrum, then from the definition we just provided
    it
    follows that for 
    $$ \pi_n(\{ X_i\}_{i\in\Z})
    \begin{cases}
        \pi_{n+k}(X_k) & k+n\geq 0 \\
        \pi_{n}(X_0) & n\geq 0 \\
        \pi_{0}X_{|n|} & n<0
    \end{cases}$$
\end{defn}
\begin{defn}[Connective spectra]
    Connective spectra are
    spectra that only have positive non-trivial homotopy groups: let $\{ X_i\}_{i\in\Z}$ be a connective
    spectra, then for all
    $n<0$ it holds that $\pi_n(\{ X_i\}_{i\in\Z})=0$. 
    
    Connective spectra form a subcategory of Sp denoted $\operatorname{Sp}_{\geq 0}$.
\end{defn}
\begin{ex}
    Many of the spectra one uses are connective:
    \begin{itemize}
        \item suspension spectra,and thus remarkably the sphere spectrum $\mathds{S}$
        \item the Thom spectrum MO
    \end{itemize}
\end{ex}

\begin{rem}[Connective spectra=infinite loop spaces]
    An infinite loop space is a 0-connective pointed space (since it is non-empty), hence it does not have
    any negative
    homotopy groups and it has exactly the same structure $X_n\simeq \Omega X_{n+1}$ as a spectrum.
    
    Thus, infinite loop spaces coincide with connective spectra.
\end{rem}
\begin{rem}[Grouplike $\E_\infty$-spaces=Picard groupoids]\label{InfLoopsPicard}
    
    Recall (\ref{EAlg}) that grouplike
    $\mathbb{E}_\infty$-spaces coincide with
    Picard $\infty$-groupoids:
    \begin{itemize}
        \item the $\E_\infty$ makes them homotopy coherent commutative
        monoid objects in $\operatorname{Grpd}_\infty$, i.e. symmetric monoidal $\infty$-groupoids
        \item since they are grouplike, they have inverses because if we first
        apply $\pi_0$, then forget about the commutative structure, and land in the category of groups, that
        means that the inverses must have been there from the start
    \end{itemize} 
\end{rem}

The following result proves the that all grouplike $\E_\infty$-spaces (aka Picard
$\infty$-groupoids) 
are connective spectra (aka infinite loop spaces). Hence, we will be able to conclude that Picard
$\infty$-groupoids coincide with infinite
loop spaces.  However, what is really important for us
is that Picard groupoids coincide with connective
spectra since this is the fact that will allow us to recognize that invertible field theories are maps of 
spectra.

Roughly, passing to the limit $n\to\infty$ in the recognition theorem for $n$-iterated loop spaces with
some abstract nonsense regarding 
the definition of the category of spectra as the sequential homotopy limit
$\operatorname{lim}(...\leftarrow\sat_\ast\leftarrow\sat_\ast\leftarrow\sat_\ast\leftarrow ...)$ leads to the 
following result.
\begin{thm}[Recognition theorem for connective spectra/infinite loop
    spaces]\label{RecognitionConnectiveSpectra}
    Connective spectra coincide with grouplike $\E_\infty$-spaces.
    
    There is an equivalence: 
    $$\mathbf{B}^{\infty}:\Alg_{\mathbb{E}_{\infty}}^{\operatorname{gp}}(\mathscr{S})\leftrightarrows\operatorname{Sp}_{\geq 0}:\Omega^{\infty}$$
    
    where $\mathbf{B}^{\infty}$ denotes the classifying spectrum functor.
\end{thm}
See \cite[7.2.2.11]{lurie2008higher},
\cite[Corollary II.24]{KtheoryHebestreitWagner} or
\cite[3.29]{NardinStable} for proofs.

We can conclude this fast-paced introduction to spectra with the following slogan:

$$\textbf{infinite loop spaces=connective spectra=Picard $\infty$-groupoids=grouplike }\E_\infty\textbf{-spaces}$$

\section{Spectra and higher algebra \extra}
In sum, spectra are a higher categorical analog of abelian groups (see \ref{Ab-Enriched Cat}), they have been in fact called “the abelian groups of the 21st century” \cite{Mazel-Gee2024}
because they provide very powerful techniques for many areas of mathematics
where there is currently much progress, e.g. in algebraic K-theory. One can do algebra with 
spectra as one does ordinary algebra with abelian groups and rings, this subject is usually called 
'higher algebra'\footnote{A
    funny synonym for higher algebra is brave new algebra, referring to Huxley's dystopian
    novel \textit{Brave New World}. It was coined by Waldhausen, one of
    the pioneers of this discipline, hinting at the fact that it seems to have myriads of intriguing
    opportunities but at the same could lead into a world of pure abstractions detached from mathematics
    with actual content. Examples of successful applications of such machinery
    to concrete problems, e.g. the counterexamples to Ravenel's telescope conjecture
    \cite{burklund2023ktheoretic}, convince the author of this section that this nightmare was avoided,
    although 
    it must be said that they are clearly biased.}. This is not just a vague analogy, 
but something that can be motivated formally, for instance, abelian groups are discrete spectra:
$$\pi_0:\operatorname{Sp}\to\operatorname{Ab}$$
where in particular $\pi_0(\mathds{S})=\Z$
%In a certain sense, $\mathds{S}$ is analgous to $\Z$ but in another category: \Z is the unit in Ab, while $\mathds{S}$ is the unit in the category of spectra. 

We make the comparison clearer via the following table\footnote{his table was inspired by a table in Thomas
    Nikolaus' talk: \url{https://www.youtube.com/watch?v=SJNKs1PxT9g}. } :

\begin{center}
    \begin{tabular}{||c|c||}
        \hline
        algebra & higher algebra\\ [0.5ex]
        \hline\hline
        set & $\infty$-groupoid\\
        \hline
        abelian group & spectrum \\
        \hline
        Ab & Sp \\
        \hline
        \Z &  $\mathds{S}$ \\
        \hline
        ring & $\E_1$-ring spectrum \\
        \hline
        commutative ring & $\mathbb{E}_\infty$-ring spectrum \\
        \hline
        Ab-enriched category & Sp-enriched category \\
        \hline
        abelian category & stable category \\
        \hline
    \end{tabular}
\end{center}
See \ref{Ab-Enriched Cat} and \ref{AbelianCat} for the definitions of Ab-enriched category and
abelian category. See \ref{EAlg} for the definitions of $\E_1$-ring spectrum and of
$\mathbb{E}_\infty$-ring spectrum. A stable
category is a pointed $\infty$-category with finite limits and colimits where $\Sigma\dashv\Omega$ do
not just form an adjunction,
but are equivalent. 
There is an equivalent definition of stable category that more closely resembles the definition of abelian
category, i.e. an additive $\infty$-category with further conditions, see \cite[Section 1]{Luriealgebra}.